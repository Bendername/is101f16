\documentclass[10pt,twoside,a4paper,openright]{report}
\usepackage[Bjornstrup]{fncychap}
% enforce latex compiler to interpret files as utf8 encoded
\usepackage[utf8]{inputenc}
% Make latex understand and use the typographic
% rules of the language used in the document.
\usepackage[danish,english]{babel}
% Use the vector font Latin Modern which is going
% to be the default font in latex in the future.
\usepackage{lmodern}
% Choose the font encoding
\usepackage[T1]{fontenc}

\usepackage{float}
\restylefloat{table}
% load a colour package
\usepackage[table,xcdraw]{xcolor}
%\usepackage{xcolor}
\definecolor{aaublue}{RGB}{33,26,82}% dark blue
% The standard graphics inclusion package
\usepackage{graphicx}
% Set up how figure and table captions are displayed
\usepackage{caption}
\captionsetup{%
  font=footnotesize,% set font size to footnotesize
  labelfont=bf % bold label (e.g., Figure 3.2) font
}
% Make the standard latex tables look so much better
\usepackage{array,booktabs}
\usepackage[flushleft]{threeparttable} % for table notes
% Enable the use of frames around, e.g., theorems
% The framed package is used in the example environment
\usepackage{framed}
%% source code listing.
\usepackage{listings}
% Defines new environments such as equation,
% align and split
\usepackage{amsmath}
% Adds new math symbols
\usepackage{amssymb}
% Use theorems in your document
% The ntheorem package is also used for the example environment
% When using thmmarks, amsmath must be an option as well. Otherwise \eqref doesn't work anymore.
\usepackage{amsmath,amssymb,framed}
\usepackage[framed,amsmath,thmmarks]{ntheorem}
% Change the headers and footers
\usepackage{fancyhdr}
\fancyhead{}
\fancyhead[LO]{\leftmark}   % Chapter will only appear on the (L)eft side of (O)dd pages.
\fancyhead[RE]{\rightmark}  % Section will only appear on the (R)ight side of (E)ven pages.
% Enable arithmetics with length. Useful when
% typesetting the layout.
\usepackage{calc}

%cross sign
\usepackage{bbding}

% Add the \citep{key} command which display a
% reference as [author, year]
%\usepackage[square]{natbib}
% Appearance of the bibliography
\bibliographystyle{plain}

%%%%%%%%%%%%%%%%%%%%%%%%%%%%%%%%%%%%%%%%%%%%%%%%
% Misc
%%%%%%%%%%%%%%%%%%%%%%%%%%%%%%%%%%%%%%%%%%%%%%%%%
% for multicolumn itemize and enumations
\usepackage{multicol}
% Add bibliography and index to the table of
% contents
\usepackage[nottoc]{tocbibind}
% Add the command \pageref{LastPage} which refers to the
% page number of the last page
% package required in order to use LastPage
\usepackage{lastpage}

\usepackage[
%  disable, %turn off todonotes
  colorinlistoftodos, %enable a coloured square in the list of todos
  textwidth=\marginparwidth, %set the width of the todonotes
  textsize=scriptsize, %size of the text in the todonotes
  ]{todonotes}

% enable english st, nd, rd, th to numbers
  \usepackage{nth}
% Enable hyperlinks and insert info into the pdf
% file. Hypperref should be loaded as one of the
% last packages
\usepackage{hyperref}
\hypersetup{%
  %pdfpagelabels=true,% Results in: "Warning: Option `pdfpagelabels' has already been used,(hyperref) - setting the option has no effect"
  plainpages=false,%
  pdfauthor={Author (s)},%
  pdftitle={Title},%
  pdfsubject={Subject},%
  bookmarksnumbered=true,%
  colorlinks,%
  citecolor=aaublue,%
  filecolor=aaublue,%
  linkcolor=aaublue,% you should probably change this to black before printing
  urlcolor=aaublue,%
  pdfstartview=FitH%
}
\newcommand\JSONnumbervaluestyle{\color{blue}}
\newcommand\JSONstringvaluestyle{\color{red}}

% switch used as state variable
\newif\ifcolonfoundonthisline

\makeatletter

\lstset{
basicstyle=\small\ttfamily,
columns=flexible,
breaklines=true
}

\lstdefinestyle{json}
{
  showstringspaces    = false,
  keywords            = {false,true},
  alsoletter          = 0123456789.,
  morestring          = [s]{"}{"},
  stringstyle         = \ifcolonfoundonthisline\JSONstringvaluestyle\fi,
  MoreSelectCharTable =%
    \lst@DefSaveDef{`:}\colon@json{\processColon@json},
  basicstyle          = \ttfamily,
  keywordstyle        = \ttfamily\bfseries,
}

% flip the switch if a colon is found in Pmode
\newcommand\processColon@json{%
  \colon@json%
  \ifnum\lst@mode=\lst@Pmode%
    \global\colonfoundonthislinetrue%
  \fi
}

\lst@AddToHook{Output}{%
  \ifcolonfoundonthisline%
    \ifnum\lst@mode=\lst@Pmode%
      \def\lst@thestyle{\JSONnumbervaluestyle}%
    \fi
  \fi
  %override by keyword style if a keyword is detected!
  \lsthk@DetectKeywords%
}

% reset the switch at the end of line
\lst@AddToHook{EOL}%
  {\global\colonfoundonthislinefalse}

\makeatother
% Enable printing of section/chapter name instead of number.
% Requires use of \nameref{<label>} instead of \ref{<label>}.
\usepackage{nameref}

% Dont know what this does, but was required for formatting tables
% as specific way
\usepackage{rotating}
\usepackage{pifont}
\newcommand*\rot{\rotatebox{90}}
\newcommand*\tilt{\rotatebox{70}}
\newcommand*\OK{\ding{51}}

% setup code listings to sane defaults
\lstset{frame=single,
    keepspaces=true,
    keywordstyle=\color{blue},
    numbers=left,
    captionpos=b,
    breakatwhitespace=true,
    breaklines=true,
    tabsize=2,
    basicstyle=\ttfamily\footnotesize}

% used to include eps files
\usepackage{epstopdf}

% This package allows changing of counters in latex
% We use it to use continuous numbering of chapters across parts.
\usepackage{chngcntr}

% Sort of replaces hyperref
\usepackage{bookmark}

% Allows nested figures
\usepackage{subcaption}

% because we want nice things... no, really just to be able to put ohm symbol.
\usepackage{gensymb}

\usepackage{pdfpages}

% Make it possible to temporarily adjust the text width
\usepackage{changepage}

\newtheorem{theorem}{Theorem}

%Anders COMMAND
\newcommand{\todoanders}[2][ ]
{
  \ifthenelse{\equal{#1}{inline}}
    {\todo[inline, size=\small, color=blue!40]{\textbf{Anders:} #2}}
    {{\includegraphics[height=2ex]{setup/Comments/anders}} \todo[size=\small, color=blue!40]{\textbf{Anders:} #2}}
}


%Benjamin COMMAND
\newcommand{\todobenjamin}[2][ ]
{
  \ifthenelse{\equal{#1}{inline}}
    {\todo[inline, size=\small, color=blue!40]{\textbf{Benjamin:} #2}}
    {{\includegraphics[height=2ex]{setup/Comments/benjamin}} \todo[size=\small, color=yellow!40]{\textbf{Benjamin:} #2}}
}

%Hax COMMAND
\newcommand{\todohax}[2][ ]
{
  \ifthenelse{\equal{#1}{inline}}
    {\todo[inline, size=\small, color=blue!40]{\textbf{Daniel:} #2}}
    {{\includegraphics[height=2ex]{setup/Comments/hax}} \todo[size=\small, color=green!40]{\textbf{Hax:} #2}}
}


%Michael COMMAND
\newcommand{\todomichael}[2][ ]
{
  \ifthenelse{\equal{#1}{inline}}
    {\todo[inline, size=\small, color=blue!40]{\textbf{Michael:} #2}}
    {{\includegraphics[height=2ex]{setup/Comments/michael}} \todo[size=\small, color=orange!40]{\textbf{Michael:} #2}}
}


%Dennis COMMAND
\newcommand{\tododennis}[2][ ]
{
  \ifthenelse{\equal{#1}{inline}}
    {\todo[inline, size=\small, color=blue!40]{\textbf{Dennis:} #2}}
    {{\includegraphics[height=2ex]{setup/Comments/dennis}} \todo[size=\small, color=purple!40]{\textbf{Dennis:} #2}}
}

%Brian COMMAND
\newcommand{\todobrian}[2][ ]
{
  \ifthenelse{\equal{#1}{inline}}
    {\todo[inline, size=\small, color=blue!40]{\textbf{Brian:} #2}}
    {{\includegraphics[height=2ex]{setup/Comments/brian}} \todo[size=\small, color=red!40]{\textbf{Brian:} #2}}
}

% Definition
\theorembodyfont{\normalfont\normalsize}        %Angivelse af tekstens udseende i "`sætningens"' >body<
\theoremheaderfont{\bfseries\large}             %Angivelse af tekstens udseende i "`sætningens"' >head<
\theoremstyle{break}                            %Angivelse af linieombrud mellem "`sætningens"' >head< og >body<

\definecolor{gray}{rgb}{0.5,0.5,0.5}
\renewcommand*\FrameCommand{{\color{gray}\vrule width 5pt \hspace{10pt}}}

\newframedtheorem{definition}{Definition}[chapter]
\newframedtheorem{theoremm}{Theorem}[chapter]