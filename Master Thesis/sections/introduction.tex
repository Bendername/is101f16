%!TEX root = ..\Main.tex
\chapter{Introduction}
In later years research within HCI has shifted towards User Experience(UX) and the actual experience with a system.
User Experience is usually analyzed through very subjective measures, such as Self-Assessment Manikin (SAM), expert analysis, think-aloud and cued recall debrief. 
All of these methods not only take a long time to perform, but are also error prone with regards to subjective analysis', such as peak-end rule and expert bias.
In later years this has made space for some interesting research surrounding objective data gathering through sensors attached to the human body, and subsequent analysis of that data to estimate a users state of mind or user experience in some form or another. 
This is interesting because of the potential rewards using objective data rather than subjective data. 
If it is possible to accurately predict a mental state, issues like peak-end and expert bias could be removed. 
Further by analysing data rather than video material and having test participant interviews could drastically reduce the amount of time required to perform a user experience evaluation. 
Most research within this area has focused solely on single sensors as data providers, and as such it would be interesting to see if any improvements could be made using multiple sensors and techniques from machine intelligence to fuse the results from these sensors. 
Usually the sensors used are also consumer graded hardware, understood as lower graded hardware which is not as expensive as medical grade hardware. 
Hardware of this quality is often more erroneous or imprecise, and potential improvements could be made in terms of using fusion to allow each individual sensor to support other sensors, and hence having a more precise classification overall.

\section{Paper 1 research questions}
The first paper explores the idea that it is possible to accurately predict a mental state. A subjective method (SAM) is used to collect subjective values for the mental state which are used as verification of the predictions based of off objective physiological data. The first hypothesis is regarding this correlation:

\textit{There is a statistically significant correlation between subjective SAM ratings and physiological measurements from consumer-grade sensors.}\\\\

The second hypothesis is that using machine intelligence fusion techniques will improve the accuracy of predicting using all sensors data together, rather than using them individually:

\textit{Statistically, fusion of consumer-grade sensors has a significantly higher prediction rate than each sensor individually.}

\section{Paper 2 research questions}
The first paper validates the premise of the second paper; that it is possible to use physiological data to predict user experience.
This validation makes further research within more subtle emotional changes a reasonable target. 
The focus in this paper is on usability problems and if it is possible to detect these.
