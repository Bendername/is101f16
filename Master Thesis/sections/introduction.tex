%!TEX root = ..\Main.tex
\chapter{Introduction}
In later years research within HCI has shifted towards User Experience(UX) and the actual experience of a system.
User Experience is usually measured and analyzed through methods using subjective measurements such as expert analysis, Think-Aloud and Cued-Recall Debrief.
Although well-established and renowned, such methods are error prone with regards to subjective analysis, leading to
phenomenons such as Peak-End rule\cite{cockburn_peakend} and memory bias.

In recent years, more research focusing on objective measurements, gathered through sensors, has occurred. 
The main idea, with the objective measurement, is to measure the human body's physiological response and subsequent apply some method to estimate an affective state or user experience based on the physiological data.
This is interesting because of the potential rewards using objective data rather than subjective data, such as reducing memory bias.

If it is possible to accurately predict a person's affective state, and apply such knowledge to the area usability testing, then issues like peak-end and evaluator effect\cite{eval_effect_research} could be reduced to a minimum.
Furthermore, points of interest in a usability test based on the physiological could yield valuable information to a third-party evaluator, as it could reduce the time required to analyse a usability test.
Most research within this area has focused solely on single sensors as data providers, and as such it would be
interesting to see if machine intelligence techniques could be used to fuse the physiological data from each sensor, to increase the precision of predicting a persons affective state.

The focus of this thesis will therefore mainly be towards the detection of a user experience through physiological data, and the appliance of that knowledge to a usability test context to examine its practicability.