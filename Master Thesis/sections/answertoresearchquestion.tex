%!TEX root = ..\Main.tex
\chapter{Answer to hypotheses and research question}
\section{Paper 1}
In this paper it was explored that it is possible to gather physiological data through sensors, and further use this data to predict subjective SAM ratings with individual sensors as well as using fusion techniques. 
Participants were subjected to stimuli in the form of IAPS pictures, presented in a self developed application. 
They also reported subjective SAM values after each stimulus. 
Synced physiological data was collected in another application, and the sensors used were GSR, EEG, Pulse Sensors as well as a Kinect.
A SVM was selected as the classification technique and fusion techniques were stacking and voting.

In paper one there was the following hypotheses:
\begin{itemize}
    \item \textbf{H1:} Physiological measurements from consumer-grade sensors using a classification technique can achieve significantly higher accuracy than naive guessing when predicting subjective SAM ratings.
    \item \textbf{H2:} Statistically, fusion of consumer-grade sensors has a significantly
higher prediction rate than each sensor individually.
\end{itemize}

\textbf{H1} was partly confirmed prediction using sensor and sensor fusion was significantly better than best case guessing in all cases beside voting on the valence 3 grouping. 
The second hypothesis propose that using machine learning fusion techniques for multiple sensors are better than using a single sensor. 
Results show that the technique ``voting'' is not substantially better than single sensors and other methods, however the technique ``stacking'' performs significantly better than most methods. This confirms the second hypothesis.

Even though the results showed that using sensors it better than random guessing, the appliance of using this over traditional methods seem to unreliable or even unfeasible at even a 3-point SAM rating.
However since we showed some ability to predict the the affective state of a person at a low resolution, applying this technique to areas which does not require the same resolution as user experience could be interesting.

\section{Paper 2}
The paper had the following research questions:\\

\textit{Is it possible to detect usability problems from physiological data gathered during testing?}\\

\textit{Could a combination of physiological data gathered from multiple sensors possibly increase the reliability of such detections?}\\

The results showed that it is possible to detect usability to some degree. 
The HR, GSR and Kinect showed reasonable result when measuring how many correctly detect usability problems in regards to much data which was falsely detected. However the EEG seemed to act more unreliable and less accurate than the other sensors.

Voting was done both conservatively and aggressively.
The conservative approach showed robust results at voting 1 and 2. Whereas the aggressive approach had considerable more noise making it useless at voting 1 but showed reasonable results at voting 2 and 3.

These results yield no conclusive results it does, however, it propose to some interesting uses for the method.
This includes assisting third-party evaluators in finding usability problems. This is due to the fact that a classifier with a low Nu value can predict points of interest with reasonable rate and low noise.
This could potentially enable the evaluators to find some of the usability problems without too much video analysis.