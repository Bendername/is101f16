\newcommand{\votinggraphs}[6]{
    \begin{figure}[h!]
    \begin{minipage}[t]{0.5\textwidth}
        \includegraphics[width=\linewidth,keepaspectratio=true]{graphics/graphs/voting/#1}
        \caption{#2}
        \label{#3}
    \end{minipage}
    \hspace*{\fill} % it's important not to leave blank lines before and after this command
    \begin{minipage}[t]{0.5\textwidth}
        \includegraphics[width=\linewidth,keepaspectratio=true]{graphics/graphs/voting/#4}
        \caption{#5}
        \label{#6}
    \end{minipage}
    \end{figure}
}

\newcommand{\easyvotinggraphs}[5]{ %y_axis caption1 label1 caption2 label2
  \votinggraphs
  {voting-False_cover_rate_(FCR)-#1-Aggressive.pdf}{#2}{#3}
  {voting-False_cover_rate_(FCR)-#1-Conservative.pdf}{#4}{#5}
}


\easyvotinggraphs{Events_hit_rate_(EHR)}
{Showing voting based on an aggressive scoring function. The lightest blue shade is 1 vote, and the darkest is 4 votes. The two shades in between are voting 2 and 3.}{fig:voting_aggresive_ehr}
{Showing voting based on a conservative scoring function. The lightest blue shade is 1 vote, and the darkest is 4 votes. The two shades in between are voting 2 and 3.}{fig:voting_conservative_ehr}


%\easyvotinggraphs{Precision}
%{Showing voting based on an aggressive scoring function}{fig:voting_aggressive_pres}
%{Showing voting based on a conservative scoring funtion}{fig:voting_conservative_pres}