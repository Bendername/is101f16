%!TEX root = ..\Main.tex
%Struktur:
	%Usability test/evaluering
	% Problem med usability test/evaluering 
	% 	emotions og user frustration
	%  Man kan måle det!
	% -> det vi vil.
	
\section{Introduction}
%http://delivery.acm.org.zorac.aub.aau.dk/10.1145/2500000/2491883/p258-liapis.pdf?ip=130.225.53.20&id=2491883&acc=ACTIVE%20SERVICE&key=36332CD97FA87885%2E1DDFD8390336D738%2E4D4702B0C3E38B35%2E4D4702B0C3E38B35&CFID=764220047&CFTOKEN=51730471&__acm__=1459164823_6fcb563aae9287be26350addc8efbe52
%artikel vi kan bruge som støtte.

%http://www.tandfonline.com.zorac.aub.aau.dk/doi/abs/10.1207/s15327590ijhc1703_3#aHR0cDovL3d3dy50YW5kZm9ubGluZS5jb20uem9yYWMuYXViLmFhdS5kay9kb2kvcGRmLzEwLjEyMDcvczE1MzI3NTkwaWpoYzE3MDNfM0BAQDA=
%Fortæller om hvad der kan skabe frustrationer hos users - well citet
%Kan bruges til hvad vi gerne vil seede af problemer

% Structure for Introduction:
    %- How is UT done now
    %- Why do UT
    %- What is fallible with the current approach
    %- How have researchers tried to mitigate those fallacies
    %    - Briefly mention various approaches
    %    - Mention Anders' work as an approach we deem to have the right "direction"
    %    - We consider our work as being future work to Anders' or an extension
    %- How to measure a user's affective state
    %    - Sensors, self-report, expert evaluation
    %- Tie the affective state "frustration" to UPs
    %    - Mention similar attempts such as Anders'
    %- State our goal - to try and tie the affective state "frustration" to UPs
    %    - Sub-goal, to tie higher levels of frustration to higher severity of UPs
    %- Anders have tried this, how is our attempt different?
    %    - from Anders' paper.discussion: "The inclusion of other measures such as heart
    %        beat may shed some light on understanding frustration induced by
    %        usability problems" - (we could \textit{hook} onto this and introduce such
    %        sensors!)
    %    -  Anders' paper especially mentions the many-to-one problem: different
    %        emotional states can result in the same physical response, e.g. GSR
    %        peak can be both \textit{negative} and \textit{positive} valence. (we
    %        can mitigate this issue by including sensor(s) that also measure
    %        valence)
    %    - Anders had to consider the evaluator effect
    %    - We include more sensors, side-step the evaluator effect by using seeded problems
    %- State briefly how we setup a test-environment (seeded problems, sensors, cued-recall debrief) to try and confirm/reach our goal
    %- Argue that a subjective measurement of affective state could considerably reduce UT complexity and errors
    %    - Argue how this is the case


Usability testing has long been an important aspect of software development and
increasingly so since the paradigm shift from usability to user experience (UX).
However, as Bruun et al.~\cite{LH-paper} mention, the approaches used to
identify usability problems (UP) and categorise their severity are still based
on traditional methods and performance based metrics such as task completion and effort required. \cite{LH-paper} suggests revised methods
for identifying UPs that are instead based on a user's affective state as
traditional methods are encumbered by several fallacies.
%[17, 34, 60, 77] = "UP defined in terms of cognetive and performance-based
%impacts that the UP excerts on the user or on developers"

% maybe mention here that our work can be considered an extension/future work to
% Anders' work (we share his sentiment/agree with him)

Traditional methods for identifying UPs focus on the time taken to complete a
given task and the effort required. Strategies such as \textit{think
aloud}~\cite{use_of_TA_and_IDA} can help observers better gauge a user's
thoughts, and to some extent affective state, while experiencing UPs, but does
not formalise capturing affective state in particular and is inherently a
subjective evaluation. Three predominant methods formalised for capturing a
user's affective state are: affective self-report, physiological reactivity and
observable behaviour~\cite{BRADLEY199449}. Strategies such as \textit{think
aloud} fall within the \textit{observable behaviour} category, while
\textit{self-report} covers methods such as \textit{Self-Assessment Manikin}
~\cite{BRADLEY199449}, both of which are subject to
subjectivity-bias. Obervable behaviours inherently because evaluators are making a subjective opinion based of off their own estimations which can vary\cite{eval_effect}, and SAM because of effects like the peak-end memory bias \cite{cockburn_peakend}. Physiological reactivity requires measuring a
user's physiology during testing and is not yet regarded as common practice
\todo{cite our own review here?}.  Physiological measuring can help reduce or
eliminate the subjectivity-bias in traditional methods, and we hypothesise that
the reason it has not yet reached wide-spread adoption is due to the lack of
reliable and formalized methods.

Bruun et al.~\cite{LH-paper} formalized a method and formula for associating
\textit{galvanic skin response} (GSR) and eye-tracking with the discrete
negative affective state \textit{frustration}\todo{cite bruun, and maybe also wtf
GSR is}. As mentioned by Bruun et al.~\cite{LH-paper}, frustration is well-studied
and manifests whenever expectations or rewards are not met in a timely manner
\todo{cite}. They found correlation between peaks found in GSR signals and
SAM-ratings, but was unable to confirm a relationship between the
\textit{severity} of a UP and the level of frustration experienced. We find the
hypothesis that frustration and UP severity are associated to be sound and aim
to confirm it using a different approach that attempts to mitigate some of the
caveats they mention, such as the \textit{evaluator effect}~\cite{eval_effect}.

% TODO: write more to intro
    %- Make link to our first article (that we can measure affective state to
%some degree based in stimuli induced by images (IAPS), using multiple sensors
%and fusing their results)
    %- Mention other caveat (multifaceted emotions, non-orthogonal A/V, that
        %Anders only measure arousal etc., that we aim to solve
    %- State briefly how we setup a test-environment (seeded problems, sensors, cued-recall debrief) to try and confirm/reach our goal
    %- Argue that a subjective measurement of affective state could considerably reduce UT complexity and errors
    %    - Argue how this is the case
\section{Related work}
Several attempts have been made to improve both efficiency and accuracy of
usability evaluation experiments. Kjeldskov and Stages' \textit{Instant Data
Analysis} (IDA) method aims to significantly reduce the effort and time required
to post-analyse data when identifying usability
problems~\cite{instant_data_analysis}. Compared to traditional video data
analysis techniques, they found that in only 10\% time they were able to
identify 85\% of critical usability problems. Jonathan J. et
al.~\cite{use_of_TA_and_IDA} found the IDA
method compelling for its reduction in time and effort required, but mentions
several shortcomings to take into consideration of using it over a traditional
method.

\textit{Think-Aloud} sessions can help evaluators extract the thought-process
and affective state of test participant while they interact with the system
under test. During think-aloud sessions, the test participant verbalizes how
they interact with the system and how they feel while doing so.  Several studies
have been made, investigating the efficiency and accuracy of think-aloud
\textit{protocols}, i.e. how the subject should be interrogated during
testing~\cite{two_think_aloud_protocols_study}. However, no matter how refined
such methods become, they are inherently subjective.

While still a relatively new and unexplored area within HCI research,
physiological sensors are increasingly being applied in usability experiments.
Elling et at.~\cite{concurrent_think_aloud_eye_tracking} investigated the
relationship between what users verbalized during think-aloud sessions and where
their gaze was at (i.e. used eye-tracking equipment), to both scrutinize the
think-aloud method and to complement it. Similarly, P\"{a}tsch et
al.~\cite{using_sensor_graphs_think_aloud} complimented think-aloud recall
sessions using GSR sensor data.

A phenomenon that can significantly impact how and which usability problems are
found and categorized, is the \textit{evaluator effect}~\cite{eval_effect}.
The evaluator effect is the phenomenon that a set of evaluators will individually only find a subset of all usability problems.
Further the evaluators will tend to find different problems, and score them differently as well.\cite{eval_effect}
Hertzum et al. found that not only do
usability experts identify substantially different usability problems, they also
disagree on how they should be categorized in
severity~\cite{eval_effect_research}. Based on their findings, they suggest that
several evaluators and domain experts should participate in evaluating critical
software.

As mentioned in the introduction, Bruun et al. combined both GSR sensor and
eye-tracking data to compliment think-aloud recall sessions. They found a
significant correlation between moments of peaks in GSR readings and instances
of usability problems. They results were however encumbered by the
\textit{evaluator effect} and conclude that a structured and operationalized
method for capturing affective state and associated it with usabilty problem
severity is still needed within HCI

It is clear from the litterature and recent researchtrends within the HCI community, that a revision of the traditional usability evaluation methods is being explored. 
Current methods rely on ''experts' and their subjective analytic skills whilst being focused on performance-based metrics.
As exemplified, more research is taking place within the area of using and applying sensors to estimate the affective state of users, however, this research is limited and usually only one sensor is used. 
Another complication is the fact that much of the hardware is expensive, or consumer graded hardware is too imprecise to yield robust results.

The focus of this paper is to use a set of consumer graded hardware to increase the overall robustness of the system as a whole. 
By using multiple sensors and appropriate machine intelligence techniques we hope to show that it is possible to use objective physiological data to accurately find usability problems as a direct alternative or extra tool to tradtional usability evaluation methods. 
As such, our hypothethis' are:\\\\

\textit{\textbf{H.1.} Physiological sensor data can be used to detect usability problems with a significantly low amount of false-positives. False-positives are the classification of a usability problem being present, while it is not.}\\\\
\textit{\textbf{H.2.} Physiological sensor data can be used to detect usability problems with
 a significantly low amount of false-negatives. False-negatives are the failure to classify a usability problem being present}\\\\
\textit{\textbf{H.3.} There is a statistical significant correlation between severity ratings and physiological measurements.}

% Structure for Related work
    %- Mention Anders' work
    %    - He has tried what we are about to, but not with additional sensors
    %- Mention work that tries to identify users' affective state in relation to UPs
    %    - Mention the fallacies in those approaches (not fusing sensors, using expert evaluation, etc.)
    %- Mention (as Anders also mentions) that affective state and UP severity has not been operationalized
    %    - We want to close that gap - bring the measurement of affective state and UP severity together
    %- Mention related work that is fallible because of the evaluator effect (measuring with sensors, but not using a seeded test-setup)
    %    - (Maybe mention Anders' work, but try not to bash it)
    %    - We can maybe more precisely confirm the hypothesis by using seeded problems and side-step the evaluator effect
    %- Mention that current approaches are fallible because of the evaluator effect
    %- Mention frustration and how research have shown how/when it manifests (when expectations are not met)
    %- Mention that we are confident that we can measure frustration, based on last semester's results
    %    - Mention that we can measure the high-arousal/negative-valence
    %    quadrant with X accuracy, which is where the basic emotion frustration is
    %    situated, according to Z and Y


%http://www.tandfonline.com.zorac.aub.aau.dk/doi/pdf/10.1080/03610730500206808
% Cite for SAM

%http://www.robots.ox.ac.uk/~davidc/pubs/NDreview2014.pdf
% Cite and use for novelty detection


%\section{Method} (this is in another file)
% Structure for Method
    %- Some text that lead up to the hypothesis
    %- Present the hypothesis
    %- Present how in broad terms we expect to be able to confirm/reject the hypothesis
    %    - Mention what we want to measure
    %    - Mention how we measure it
    %    - Mention experiment setup
    %    - Mention that we use multiple sensors, fuse their results for input to some MI approach (decide which approach we are going to use)
    %- Subsection: experiment setup
    %    - How it is setup, sensors, what we measure, participants, seeded problems software, procedure/protocol, cued-recall-debrief (from parallel research)

% Structure for Future work
    %(This depends heavily on the results we get, but we assume accepting our hypothesis/success)
    %- Establish a severity scale based on affective state, not considering completion time or effect as they are a product of affective state
    %- Future work should based experiments on "real software", but consider the evaluator effect (maybe test against it)

%\begin{itemize}
%    \item traditional view on UP and severity scales might be fallible/lacking,
%        as they do not consider the aspect of affect/emotional state
%    \item traditional views on UP is mainly focused on time, effort and completion
%        time
%    \item affective state is rarely mentioned in severity scales. Can we maybe
%        do this? (idea: develop a new severity scale based on affective state)
%    \item even though HCI has shifted focus from usability to UX, few attempts have
%        been made to do the same with severity ratings (base them on an
%        \textit{experience}, i.e. affective state)
%    \item Three systems for measuring emotional response/state:
%        \begin{itemize}
%            \item affective self-report (SAM)
%            \item Physiological reactivity (sensors)
%            \item Observable behaviour (expert ratings)
%        \end{itemize}
%    \item explain/mention what could be fallible/error-prone with SAM and expert
%        ratings (its subjective)
%    \item using sensors is not common-place (could we maybe use sensors to
%        measure the affective state?)
%    \item \textit{frustration} is a measure/label for a \textit{negative
%        affective} state.
%    \item Frustration is well studied: occurs when a \textit{need} or
%        \textit{reward} is not satisfied
%    \item Frustration does not appear in severity scales (except a single instance mentioned in Anders' paper, but is not tied to physiological measurements)
%    \item Anders' paper proposed the hypothesis that users are more negatively
%        aroused when experiencing UP of higher severity, but could not confirm
%        it (we could \textit{hook}
%        onto this and try and see if we could confirm the hypothesis)
%    \item maybe explain why understanding/ordering UP severity levels are important
%        (higher levels should be prioritized to be fixed)
%    \item from Anders' paper.discussion: "The inclusion of other measures such as heart
%        beat may shed some light on understanding frustration induced by
%        usability problems" - (we could \textit{hook} onto this and introduce such
%        sensors!)
%    \item From Anders' paper "... The ability of GSR to measure arousal and thus emotion... ". (I'm not sure we agree with him, but we can add that if we include the other dimension, valence, then we get close to be able to measure emotions, at least emotional state)
%    \item Several studies confirm that GSR change when users experience stress and negative affect
%    \item Anders' paper mentions three big issues with GSR measurements (we should discuss/try to mitigate them as well)
%    \item Anders' paper especially mentions the many-to-one problem: different
%        emotional states can result in the same physical response, e.g. GSR
%        peak can be both \textit{negative} and \textit{positive} valence. (we
%        can mitigate this issue by including sensor(s) that also measure
%        valence)
%    \item Anders' paper tries to mitigate the evaluator effect. (we can sidestep this by using a seeded setup)
%    \item Anders introduces a \textit{Frustration score/metric}. It relies on Dominance as well, so if we use it, we would have to adapt it. Maybe discuss this with Anders? Its: \[F = A * [(10-V) + (10-D)]\]
%
%    \item Anders did \textit{not} find a correlation between GSR peaks and
%        severity level. If this becomes our aim, then we should try and tackle
%        some of the suggestions he presents to why he was unable to do so
%        (first occurrence, retain occurrences of the same issue, users might
%        ignore a problem to remain calm, etc.). He mentions the evaluator
%        effect being a possible cause, but we can remove this concern. Users
%        might misinterpret SAM ratings, we can maybe sidestep this concern by
%        relying on CRD.
%\end{itemize}
%
%Ideas for hypothesis/research area:
%\begin{itemize}
%    \item Solidify/establish a convincing method for measuring frustration (compared to Anders' approach, we can introduce valence and thereby measure negative affect. First paper can be used as 'proof'/confirmation that we can measure negative affect, and thereby frustration)
%    \item Attempt to find a correlation between frustration and UP severity level (should be easier in our case, now that we also measure valence)
%    \item Propose a new method for measuring/establishing UPs (by using and fusing sensors. In this method, UPs are defined by negative affect)
%    \item Propose a method for mapping negative affect to UP severity (we can use Anders' frustration metric)
%    \item Propose/formulate a new severity scale that is defined by/base on affective state (instead of cosmetic, minor, critical, then measure UPs by how much it frustrates a user. UPs that highly frustrate users should be prioritized fixing. Using this metric/scale, it is no longer important what \textit{kind} a problem is but that it frustrates the user. This is much more inline with UX in general. Identifying UPs is \textit{still} an \textit{unsolved} problem - is it because of a wrong approach? Is it better to find them using affective state? Etc., etc.)
%    \item Propose a framework/workflow/method for easily setting up lab experiments for measuring negative affect (that is, an easily understood and afforable method; focus on consumer-grade hardware and shelf-solutions for interpreting the results; focus on what kind of sensors are necessary and why, some are best at measuring A, others V; suggest easy calibration methods; etc., etc.)
%\end{itemize}
