%!TEX root = ..\Main.tex
	%Intro <- Hvad er det vi vil?
				%Vi skal ikke mudre billedet til. (ren focus på hvad artiklen vil)

\section{Introduction}
%http://delivery.acm.org.zorac.aub.aau.dk/10.1145/2500000/2491883/p258-liapis.pdf?ip=130.225.53.20&id=2491883&acc=ACTIVE%20SERVICE&key=36332CD97FA87885%2E1DDFD8390336D738%2E4D4702B0C3E38B35%2E4D4702B0C3E38B35&CFID=764220047&CFTOKEN=51730471&__acm__=1459164823_6fcb563aae9287be26350addc8efbe52
%artikel vi kan bruge som støtte.

Usability testing has long been an important aspect of software development, and according to Rubin et al.~\cite{rubin2008handbook} usability testing is user centered tests that focuses on three main groups.
Informing Design, which concerns the usefulness and learning rate of a program,
elimination of design problems and frustration, which not only focuses on removing design problems and bugs, but is also about establishing a good relation to the customer, and lastly improving profitability, which is centred around reducing maintenance and increasing sales.
However, the second group, elimination of design problems and frustration, is often measured with methods involving third party observers.
These observers has to, as objectively as possible, note if a test person is encountering a usability problem.
However, such observations are often still very subjective given the evaluator effect\cite{eval_effect}, which states that the results from the evaluations are different from individual to individual.
Strategies such as \textit{think aloud}~\cite{use_of_TA_and_IDA} can help observers better gauge a user's
thoughts, and to some extent affective state, while experiencing usability problems. However it does
not formalise capture of the affective state in particular and is inherently a third-party
subjective evaluation. 

There are three predominant methods formalised for capturing a user's affective state: affective self-report, physiological reactivity and
observable behaviour~\cite{BRADLEY199449}. Strategies such as \textit{think
aloud} fall within the \textit{observable behaviour} category, while
\textit{self-report} covers methods such as \textit{Self-Assessment Manikin}(SAM)\todo{Fucking explain this shit god fucking damnit.}
~\cite{BRADLEY199449}, both of which suffers from subjectivity-bias.
Observable behaviors inherently so, because third party evaluators are giving subjective opinions based off their own estimations and analysis, which can vary due to the evaluator effect, and SAM because of effects like the peak-end memory bias \cite{cockburn_peakend}. The peak-end memory bias is the concept that when a user uses a system, it will be the most dominant experience, good or bad, and the last experience which will be remembered.
Physiological measuring has shown at several occasions that it can, to some degree, be used to predict a persons
affective state, and thus has interesting propositions to offer to the traditional evaluation methods~\cite{eeg_facial_expressions,fusion4,90_percent_eeg_emotion}.

An example of such study was made by Bruun et al.~\cite{LH-paper}, who made a usability case-study based on a website.
A number of usability problems were found in a prior empirical study, from which they created three tasks for the participants to complete.
They recorded physiological data using \textit{galvanic skin response} (GSR) which measures sweat \cite{gsr_calibration}, and an eye-tracker detecting the gaze of the participant.
They formalized a method and a formula for associating the physiological data with the discrete negative affect state \textit{frustration}.
As mentioned in multiple studies, e.g.~\cite{LH-paper,frustration_with_computers}, frustration is well-studied
and manifests whenever expectations or rewards are not met in a timely manner, or when a goal is compromised in one way or another.
These traits match the signature of a usability problem.
Bruun et al. also had cued-recall debrief sessions, where participants reviewed video clips found from GSR peaks, and filled out SAM scales relating to the clip. 
They found a correlation between peaks in GSR and SAM-ratings, but was unable to confirm a relationship between the
\textit{severity} of a usability problem and the level of frustration experienced.
While they found a correlation, the entire study still relies on the preliminary assumption that the usability problems found by
third-party evaluations on the website in question are correctly identified.
As mentioned, such evaluations suffer from various problems, and are generally only expected to find a subset of the usability problems present in a system. 

Our focus in this study will be to use multiple physiological sensors to detect usability problems.
This can have a potential impact as another tool for third party evaluators, by finding points of interest based on physiological data. 
This would effectively be data from the primary source, the user, rather than a secondary source in terms of the evaluator who tries to describe the users experience through obsservation and analysis. 
We aim to use consumer graded sensors in order to make our setup more accessible, should other researchers find the interest to reproduce
it. We couple physiological data from various sensors with established Machine Learning(ML) techniques as the method for
detecting usability problems.