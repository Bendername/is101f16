%!TEX root = ..\Main.tex
%Struktur:
	%Usability test/evaluering
	% Problem med usability test/evaluering 
	% 	emotions og user frustration
	%  Man kan måle det!
	% -> det vi vil.
	
\section{Introduction}
%http://delivery.acm.org.zorac.aub.aau.dk/10.1145/2500000/2491883/p258-liapis.pdf?ip=130.225.53.20&id=2491883&acc=ACTIVE%20SERVICE&key=36332CD97FA87885%2E1DDFD8390336D738%2E4D4702B0C3E38B35%2E4D4702B0C3E38B35&CFID=764220047&CFTOKEN=51730471&__acm__=1459164823_6fcb563aae9287be26350addc8efbe52
%artikel vi kan bruge som støtte.

%http://www.tandfonline.com.zorac.aub.aau.dk/doi/abs/10.1207/s15327590ijhc1703_3#aHR0cDovL3d3dy50YW5kZm9ubGluZS5jb20uem9yYWMuYXViLmFhdS5kay9kb2kvcGRmLzEwLjEyMDcvczE1MzI3NTkwaWpoYzE3MDNfM0BAQDA=
%Fortæller om hvad der kan skabe frustrationer hos users - well citet
%Kan bruges til hvad vi gerne vil seede af problemer

\begin{itemize}
    \item traditional view on UP and severity scales might be fallible/lacking,
        as they do not consider the aspect of affect/emotional state
    \item traditional views on UP is mainly focused time and effort and completion
        time
    \item affective state is rarely mentioned in severity scales. Can we maybe
        do this? (idea: develop a new severity scale based on affective state)
    \item even though HCI has shifted focus from usability to UX, few attempts have
        been made to do the same with severity ratings (base them on an
        \textit{experience}, i.e. affective state)
    \item Three systems for measuring emotional response/state:
        \begin{itemize}
            \item affective self-report (SAM)
            \item Physiological reactivity (sensors)
            \item Observable behaviour (expert ratings)
        \end{itemize}
    \item explain/mention what could be fallible/error-prone with SAM and expert
        ratings (its subjective)
    \item using sensors is not common-place (could we maybe use sensors to
        measure the affective state?)
    \item \textit{frustration} is a measure/label for a \textit{negative
        affective} state.
    \item Frustration is well studied: occurs when a \textit{need} or
        \textit{reward} is not satisfied
    \item Frustration does not appear in severity scales (except a single instance mentioned in Anders' paper, but is not tied to physiological measurements)
    \item Anders' paper proposed the hypothesis that users are more negatively
        aroused when experiencing UP of higher severity, but could not confirm
        it (we could \textit{hook}
        onto this and try and see if we could confirm the hypothesis)
    \item maybe explain why understanding/ordering UP severity levels are important
        (higher levels should be prioritized to be fixed)
    \item from Anders' paper.discussion: "The inclusion of other measures such as heart
        beat may shed some light on understanding frustration induced by
        usability problems" - (we could \textit{hook} onto this and introduce such
        sensors!)
    \item From Anders' paper "... The ability of GSR to measure arousal and thus emotion... ". (I'm not sure we agree with him, but we can add that if we include the other dimension, valence, then we get close to be able to measure emotions, at least emotional state)
    \item Several studies confirm that GSR change when users experience stress and negative affect
    \item Anders' paper mentions three big issues with GSR measurements (we should discuss/try to mitigate them as well)
    \item Anders' paper especially mentions the many-to-one problem: different
        emotional states can result in the same physical response, e.g. GSR
        peak can be both \textit{negative} and \textit{positive} valence. (we
        can mitigate this issue by including sensor(s) that also measure
        valence)
    \item Anders' paper tries to mitigate the evaluator effect. (we can sidestep this by using a seeded setup)
    \item Anders introduces a \textit{Frustration score/metric}. It relies on Dominance as well, so if we use it, we would have to adapt it. Maybe discuss this with Anders? Its: \[F = A * [(10-V) + (10-D)]\]

    \item Anders did \textit{not} find a correlation between GSR peaks and
        severity level. If this becomes our aim, then we should try and tackle
        some of the suggestions he presents to why he was unable to do so
        (first occurrence, retain occurrences of the same issue, users might
        ignore a problem to remain calm, etc.). He mentions the evaluator
        effect being a possible cause, but we can remove this concern. Users
        might misinterpret SAM ratings, we can maybe sidestep this concern by
        relying on CRD.
\end{itemize}

Ideas for hypothesis/research area:
\begin{itemize}
    \item Solidify/establish a convincing method for measuring frustration (compared to Anders' approach, we can introduce valence and thereby measure negative affect. First paper can be used as 'proof'/confirmation that we can measure negative affect, and thereby frustration)
    \item Attempt to find a correlation between frustration and UP severity level (should be easier in our case, now that we also measure valence)
    \item Propose a new method for measuring/establishing UPs (by using and fusing sensors. In this method, UPs are defined by negative affect)
    \item Propose a method for mapping negative affect to UP severity (we can use Anders' frustration metric)
    \item Propose/formulate a new severity scale that is defined by/base on affective state (instead of cosmetic, minor, critical, then measure UPs by how much it frustrates a user. UPs that highly frustrate users should be prioritized fixing. Using this metric/scale, it is no longer important what \textit{kind} a problem is but that it frustrates the user. This is much more inline with UX in general. Identifying UPs is \textit{still} an \textit{unsolved} problem - is it because of a wrong approach? Is it better to find them using affective state? Etc., etc.)
    \item Propose a framework/workflow/method for easily setting up lab experiments for measuring negative affect (that is, an easily understood and afforable method; focus on consumer-grade hardware and shelf-solutions for interpreting the results; focus on what kind of sensors are necessary and why, some are best at measuring A, others V; suggest easy calibration methods; etc., etc.)
\end{itemize}
