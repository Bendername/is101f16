%!TEX root = ..\Main.tex
	
\section{Introduction}
%http://delivery.acm.org.zorac.aub.aau.dk/10.1145/2500000/2491883/p258-liapis.pdf?ip=130.225.53.20&id=2491883&acc=ACTIVE%20SERVICE&key=36332CD97FA87885%2E1DDFD8390336D738%2E4D4702B0C3E38B35%2E4D4702B0C3E38B35&CFID=764220047&CFTOKEN=51730471&__acm__=1459164823_6fcb563aae9287be26350addc8efbe52
%artikel vi kan bruge som støtte.

Usability testing has long been an important aspect of software development.
However, as Bruun et al.~\cite{LH-paper} mention, the approaches used to
identify usability problems (UP) and categorise their severity are still based
on traditional methods and performance based metrics such as task completion and effort required. \cite{LH-paper} suggests revised methods
for identifying UPs that are instead based on a user's affective state as
traditional methods are encumbered by several fallacies.
%[17, 34, 60, 77] = "UP defined in terms of cognetive and performance-based
%impacts that the UP excerts on the user or on developers"

% maybe mention here that our work can be considered an extension/future work to
% Anders' work (we share his sentiment/agree with him)

Traditional methods for identifying UPs focus on the time taken to complete a
given task and the effort required. Strategies such as \textit{think
aloud}~\cite{use_of_TA_and_IDA} can help observers better gauge a user's
thoughts, and to some extent affective state, while experiencing UPs, but does
not formalise capturing affective state in particular and is inherently a third-party
subjective evaluation. Three predominant methods formalised for capturing a
user's affective state are affective self-report, physiological reactivity and
observable behaviour~\cite{BRADLEY199449}. Strategies such as \textit{think
aloud} fall within the \textit{observable behaviour} category, while
\textit{self-report} covers methods such as \textit{Self-Assessment Manikin}
~\cite{BRADLEY199449}, both of which suffers from 
subjectivity-bias. Observable behaviors inherently because third-party evaluators are making a subjective opinion based of off their own estimations and analysis' which can vary\cite{eval_effect}, and SAM because of effects like the peak-end memory bias \cite{cockburn_peakend}.
Physiological measuring has been showed at several occasions that it can to some degree be used to reflect a persons affective state, and thus has interesting propositions to offer to traditional methods.

Bruun et al.~\cite{LH-paper} made a usability case-study of ``Danmarks Statistik (DST)''.
A number of UP's were found in an independent emperical study, for which they created a main task with three sub tasks for participants to complete.
They recorded physiological data using a \textit{galvanic skin response} (GSR) which measures sweat \cite{gsr_calibration}, and an eye-tracker detecting the gaze of the participant.
They formalized a method and a formula for associating the physiological data with the discrete negative affect state \textit{frustration}.
As mentioned in multiple studies, e.g. ~\cite{LH-paper}, ~\cite{frustration_with_computers}, frustration is well-studied
and manifests whenever expectations or rewards are not met in a timely manner or conflict when a goal is compromised in one way or another.
They had a cued-recall debrief session where the participants reviewed video clips found from GSR peaks, and filled out SAM scales relating to the clip. 
Bruun et al. found a correlation between peaks found in GSR signals and SAM-ratings, but was unable to confirm a relationship between the
\textit{severity} of a UP and the level of frustration experienced.

While Bruun et al. found a correlation, the entire study still relies on the preliminary assumption that the UPs found by third-party evaluations on DST are correct. 
As mentioned such evaluations suffers from various problems, and are generally only considered to find a subset of the UPs present in a system. 
With recent advances in consumer graded hardware and machine intelligence, and the fact that studies show correlations between UPs and measured physiological responses, we propose another approach where the usage of sensors and physiological data could be used to find UPs.
This could potentially elimate some of the need for third-party expert evaluations and not only save time, but also mitigate implications such as the \textit{evaluator effect}\cite{eval_effect}.
We define the preliminary research question as:\\

\textbf{RQ}. \textit{Is it possible to use physiological data from consumer graded hardware and machine learning, to
  eliminate some of the implications from third-party evaluations, and in usability evaluations?}

% TODO: write more to intro
    %- Make link to our first article (that we can measure affective state to
%some degree based in stimuli induced by images (IAPS), using multiple sensors
%and fusing their results)
    %- Mention other caveat (multifaceted emotions, non-orthogonal A/V, that
        %Anders only measure arousal etc., that we aim to solve
    %- State briefly how we setup a test-environment (seeded problems, sensors, cued-recall debrief) to try and confirm/reach our goal
    %- Argue that a subjective measurement of affective state could considerably reduce UT complexity and errors
    %    - Argue how this is the case