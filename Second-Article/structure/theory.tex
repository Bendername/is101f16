%!TEX root = ..\Main.tex
\section{Theory}
\todo{write what it is we are elaborating on in this section}

\subsection{Measuring UX physiologically}
Usability testing has a long history, and as mentioned, is usually focused on performance metrics, and less so
on experiential metrics such as emotions.  The practices and limitations of this has been covered in numerous articles
(e.g. \cite{usability_eval} and \cite{eval_effect}) and books (e.g. \cite{guide_to_upeval}) and the concept ``usability
problem'' is well defined.

It can be argued, that there is a fundamental intuitive difference between UPs and user-experience problems(UXP).  Where
identifying usability problems is focused on task completion and effort required, identifying UXP is focused on the
users affective state, or experience with the task.  How to measure this ``experience'' is another problem completely,
and still non-trivial. While this can be observed/collected through third party measures
(think aloud, expert evaluations, etc.), it is a subjective and fallible measure.  There is, however,
still quite some debate in this area between researchers and the exploration of this is beyond the scope of this paper,
but an overview is given by Law et. al \cite{attitudes_ux_measure}.  In Roto et al. \cite{what_is_ux} they write ``..No
generally accepted overall measure of UX exists, but UX can be made assessable in many different ways.'' which 
emphasize the problem with measuring UX, and by extension, UXPs.

To our knowledge, no conclusive research exists regarding how UXPs unfolds on a physiological level. This includes
fundamental aspects such as how experiencing UPs is reflected on the physiological level, how different (severity, nature,
duration, etc.) UPs affect this and how the order of experienced UPs might influence this as well.

While some work have dealt with single sensors such as \cite{mind_the_gap} \cite{LH-paper}, the basic extraction of
interesting physiological responses are based on subsequent expert analysis of the data.  While these interesting parts
of a physiological response may to some extent be verified against cued-recall of the same points in time, it is
inherently biased and suffers from the concept of the evaluator effect. Often, evaluators analyze data based on
intuition, e.g. finding anomalies in GSR data such as unexpected peaks, which does not guarantee that all UPs have been
extracted from the data.  This lack of a formalized method for how user-experience problems unfolds itself
physiologically over time, makes any definition of how to detect anomalies in data which could be a manifestation of
experiencing a UP, an educated guess from the researchers.

Due to limitations and the scope of this study, we will not attempt to formalize a solution to the problems outlined
above, but will take them into account when outlining the definitions used in this paper to describe and detect UPs.

\subsection{Relation between psychology and physiology in experiences}
In this section, we will try and define the relation between experiences, i.e. affective states, and physiology. We find
this understanding important, as our goal is to predict the former based on the latter.

Before defining how we interpret user experiences, let us remind ourselves how usability problems can be
defined. Lavery D. et al.~\cite[p. 254]{comp-eval-methods} suggests the following definition:

\begin{quotation}
  \textit{``A usability problem is an aspect of the system and/or a demand on the user which makes it unpleasant,
    inefficient, onerous or impossible for the user to achieve their goals in typical usage situations''} 
\end{quotation}

From this definition, we find it reasonable to expect test participants to experience affective states such as
\textit{frustration, irritation} or \textit{stress}. We also find this similar to suggestions by Brunn et
al.~\cite{LH-paper} and Lazar et al.~\cite{frustration_with_computers}, with \textit{frustration} being a predominant
affective state. Similarly, Ceaparu et al.~\cite{determining-causes-end-user-frust} claim that \textit{frustration} is
universally experienced by computer users, whenever expectations are not fulfilled and unexpected time delays are
imposed.

How such affective states manifests themselves in physiological measurements, we deem to be an advanced and complex
topic on its own. The problem can be outlined as identifying an affective state, in this case frustration, from
physiological data, i.e. as a function from the physiological domain to the psychological domain.  As mentioned by
Liapis et al.~\cite{fusion4}, Cacioppo et al.~\cite[p. 8-9]{handbook-psychophysiology} offers five plausible
connections, or relations, between the psychological and physiological domains, briefly stated as:

\begin{itemize}[noitemsep, nolistsep]
\item \textbf{one-one:} a psychological event is associated with one and only on physiological event, and vice versa
\item \textbf{one-many:} a psychological event is associated with several physiological events. The opposite could also
  hold true, that several psychological events could cause the same physiological event.
\item \textbf{many-many:} a particular set of psychological events are related to a particular set of physiological
  events
\item \textbf{null:} that there exists no particular relation between physiological and psychological events.
\end{itemize}

According to Cacioppo et al., is could even be the case that one relation holds true for one individual, while a
different relation holds for another. We suspect the above to be one of the reason why we have been unable to find
definitive and formalized methods that identify affective state based on physiology. That is not to say that related
research have not been successful in doing so - quite the contrary - but only that results vary, methods are disparate
and varies depending on the context.
\todo{list the sources/cites of a few studies with good results w.r.t. identify affective state}

If the \textbf{one-one} relation from those suggested by Cacioppo et at. holds, then some particular affective state
would always manifests itself in the same physiological event, e.g. fear would always see the same increase in
heart-rate, change in skin conductance, and the same patterns in brain-waves. We find this view unplausible, at least in
our context, and we are more inclined to accept the premise that a \textbf{one-many} relation exists. We see this as
meaning that a particular psychological event can manifest itself into several different physiological events. However,
we do expect to see similarities in those physiological events, if originating from the same psychological event, e.g
feeling \textit{fear} leads to some increase in heart-rate, variance in skin conductance, and activity from particular
areas of the brain (specific sensors on the EEG). In this work, we also assume the above to be the case across all
individuals participating in our experiments.

\subsection{User experience problem definition}
Questions such as ``when does a particular affective state manifest itself'' and ``for how long...'' when considering a
continuous stream of physiological measurements from various different types of sensors, are difficult to answer. Not
only are they difficult to answer, but the answer might also vary significantly from person to person. Our previous work
was faced with with similar questions~\cite{9th_semester_project}, however the problem was easier to approach since
stimuli was very specific and induced at very specific moments, leading us to approach the above differently.

In this definition, we consider particular moments within the experiment to be \textit{normal}, where we expect test
participants to exhibit physiological states that are \textit{not} related to \textit{frustration} or similar negative
affective states. We refer to such \textit{normal} states as the \textit{baseline}. Further, we do not expect test
participants to exhibit any positive affective states during the experiment, such as \textit{happiness} or \textit{joy}
- our test setup is designed with this in mind. Based on these assumptions, we hypothesize that any physiological
measurements that deviate from \textit{normal} could be interpreted as negative affective state. This of course leads to
the question of how do we consider some particular measurements to be outside of this \textit{normal} state? The answer
depends on the particular sensor in question, and for this we lean on our previous work.

We consider affective state changes in GSR to manifest themselves 2-4 seconds after the experience occurred which
induced the stimuli, and is noticeable within a time-frame of 3-4 seconds thereafter. EEG is considerably different,
manifesting after 350 milliseconds, lasting 760 milliseconds. Reactions to stimuli can be seen manifesting in heart-rates
after 4 seconds, and lasts for 3 seconds. Lastly face changes has a delay of 500ms and lasts for 500ms. As mentioned,
the above is based on our previous work, which in turn is based on related research within HCI, psychology and
physiology of the human body~\cite{9th_semester_project}.
\todo{cite also 1st. paper}

Based on the above, we consider \textit{windows}, i.e. samples of continuous measurements within those boundaries, for
each sensor, for a particular moment in time. We do so for moments within what we expect to be \textit{normal} state and
likewise for \textit{unknown} moments. If those \textit{unknown} moments deviate from \textit{normal}, we expect that to
indicate that the test participant is \textit{frustrated}. The above is based on many assumption, both about human
physiology and how experiencing usability problems manifests themselves in said physiology. This is necessary in order
to formalize a method with which we can conduct experiments.

In our setup, the particular moments which we deem to be \textit{normal} are defined as the two first tasks the test
participant is asked to solve, i.e. all physiological measurements collected within those moments are treated as a
\textit{baseline}. This is because the two first tasks are always chosen from a set of tasks containing no seeded
usability problems. From the baseline, we extract various statistical measurements from which \textit{unknown} data
should deviate from, in order to signify the existence of a usability problem.

\todo{I think we need a section/table below this, where we concretise the above}