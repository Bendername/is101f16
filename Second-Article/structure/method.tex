%!TEX root = ..\Main.tex


\section{Method}
In order to reject or confirm our hypotheses, a test was created.
Test participants were exposed of a program with seeded usability problems and physiological data was collected using different sensors while the system was in use. 
The collected data was subsequently used to train a classifier\todo{SVM?} which can be used to classify usability problems from normal use.\todo{Maybe change to something not classifier specific but more outlier specific.}

\subsection{Test program}
A software application was developed specifically for this test. 
Jussi P.P. Jokinen\cite{workplace_up_study} had test participants complete tasks in four every-day work applications (text editing, image processing, browser usage and presentation editing). 
Lazar et al.\cite{frustration_with_computers} found that text-processing and email related tasks had the highest amount of frustrating experiences.
Since our goal is to have a familiar test setting with seeded problems, we have elected for an email client as the application of choice.

The application was built upon a self developed framework which facilitated seeding usability problems and creating a set of tasks for the user to complete. 
An attempt was made to keep the application simple and low on functionality, to decrease the risk of unintentionally introducing usability problems, but still having the basic functionalities of a normal email application.  
The UPs associated with a specific task were only active when the task was active. 
For instance a particular task involved attaching an image to an email. 
While the task is active, the attachment process will fail three times and be successful on the fourth. 
Had any other task been active the attachment process would not contain the seeded usability problem.
The program had a total of 11 tasks of which 7 contained a seeded usability problem. 
All tasks were randomized for each test participant beside the first two which contained no seeded problems, and were used for baseline training data for outlier detection. 
Each task and their associated usability problem can be seen in the below. 
Task 3 and 7 can be considered full stoppers, and are incompletable. 
Tasks 1, 2, 4, 5 and 6 can be completed given persistence or explorative user behaviour, e.g. task 6 will in fact remove the contact, but to see it has done so requires the user to close the window and reopen it to validate it.
Each task, and potentially seeded problem in each, is shown in Table~\ref{tab:ups-desc}.

\begin{table}[h]
  \centering
  \begin{tabular}[c]{|l|p{80pt}|p{80pt}|}
    \hline
    task name                     & description                                                                                                                                    & seeded problem                                                                                           \\ \hline
    \small{Add attachement}       & \small{The user has to add an attachment to a mail}                                                                                            & \small{Program ``hangs'' for 2 seconds three times, before the attachment can be completed.}             \\ \hline
    \small{Add contact}           & \small{The user has to add a new contact to the contacts catalogue}                                                                            & \small{The ``Add Contact'' button will not work for the first three clicks}                              \\ \hline
    \small{Send Draft}            & \small{The user has to find a draft, either by creating a mail and drafting it or selecting a pre-created draft, and send it}                  & \small{An exception will show when they try to open the draft, making it impossible to send}             \\ \hline
    \small{Create a draft}        & \small{The user has to create a draft with the body: ``Rød grød med føde''}                                                                    & \small{The keyboard layout changes to American, making it impossible to type the Danish character ``ø''} \\ \hline
    \small{Write a mail}          & \small{The user has to create a mail with the body: ``Hi, my name is x and I am participating in a usability test''}                           & \small{At random intervals the caret will move while writing the mail}                                   \\ \hline
    \small{Remove Contact}        & \small{The user has to remove a specific contact from the contacts catalogue}                                                                  & \small{When clicking ``Delete'', the entire window will change to a black box}                           \\ \hline
    \small{Write mail 2}          & \small{The user has to write a mail with the body text ``Hello, I am having a birthday party 10 days from now, and this is your invitation!''} & \small{The window for writing a mail is unavailable, and the title changes to ``Not responding...''}     \\ \hline
    \small{Send a mail}           & \small{The user has to send a mail with any text, to two contacts}                                                                             & \small{None}                                                                                             \\ \hline
    \small{Save a draft}          & \small{The user has to create a mail, and draft it}                                                                                            & \small{None}                                                                                             \\ \hline
    \small{Reply to mail}         & \small{The user has to reply to a mail}                                                                                                        & \small{None}                                                                                             \\ \hline
    \small{Write and delete mail} & \small{The user has to write a mail containing any text, draft it and then delete it}                                                          & \small{None}                                                                                             \\ \hline
  \end{tabular}
  \caption{Usability problems descriptions}
  \label{tab:ups-desc}
\end{table}

\subsection{Hardware}
The hardware used for the experiment is an Emotiv Epoc~\cite{emotiv_epoc_website} for Electroencephalograph (EEG) to recording brain activity, a Mindplace Thoughtstream~\cite{thoughtstream} for Galvanic Skin Response (GSR), an Arduino with a pulse-sensor~\cite{pulsesensor} with modified software~\cite{pulsesensorgit} to measure heart rate (HR) and a Kinect V2\cite{kinect_specs3} for tracking facial changes.
The pulse-sensor software was modified to send beats per minute (BPM), inter-beat interval (IBI) and raw signal every 20 ms.
All devices are low-cost consumer grade hardware.

\subsection{Participants}
A total of 29 people participated in the test (16 male, age XX-XX SD X, 13 female age YY-YY SD Y).
The participants were students recruited from multiple faculties at Aalborg University.

\subsection{Setup}
The tests were conducted from the 13th of April, 2016, to the 30th of April 2016 and from 8:00 to 16:00 every day. 
The participants were asked to fill out a consent form prior participation. 
The participants were instructed in how to use the test program which included how to see the tasks, and how to indicate whether or not they could complete the given task. 
Before starting the test all hardware was attached to the participant and verified in terms of connectivity. 
The EEG was connected to the head according to the 10-20 system\cite{eeg_tech_10_20}, the GSR and pulse sensor was attached to their non-dominant hand. 

\subsection{Test procedure}
The test starts with an application the mail client simulation.
In the bottom right corner of the application a ``task'' window is located, which contains information regarding the test participants current task. 
A red and green button indicated that you have ``completed'' or ``not completed'' a task.  
Each task had an indefinite time threshold, but the participant could choose to continue to the next task at any given moment by pressing either a green or a red button on the keyboard. 
The participants were not allowed to talk to the test conductor during the test, unless it was absolutely necessary. 
In case any communication had to take place, it would be done through a microphone and speakers.


\section{Theory}

\subsection{Measuring UX physiologically}
Usability tests has been done for a long time, and as mentioned they usually focus on performance metrics, and less so
on experiential metrics such as emotions. 
The practices and limitations of this has been covered in numerous articles (e.g. \cite{usability_eval} and \cite{eval_effect}) and books (e.g. \cite{guide_to_upeval}) and the concept ``usability problem'' is well defined.
There is, however, a fundamental intuitive difference between UPs and user-experience problems(UXP). 
Where the usability problem is focused on task completion and effort required, the UEP is focused on the users affective state, or experience with the task.
How to measure this ``experience'' is another problem completely.
While this can be observed/collected through third party measures (think aloud, expert evaluations etc), it has always been a low resolution, subjective measure. 
There is, however, still quite some debate in this area between researchers, and the exploration of this is beyond the scope of this paper, but an overview is given by Law et. al \cite{attitudes_ux_measure}. 
In Roto et al. \cite{what_is_ux} they write ``..No generally accepted overall measure of UX exists, but UX can be made assessable in many different ways.'' which precisely emphasize the problem with measuring UX. 

There is no conclusive literature regarding how UXPs unfolds on a physiological level, and no models that accurately depict the evolution of a user-experience problem. 
While some papers have dealt with single sensors such as \cite{mind_the_gap} \cite{LH-paper}, the basic extraction of interesting physiological responses are based upon expert analysis of the data.
While these interesting parts of a physiological response may to some extent be verified against cued-recall of the same points in time, it is inherently biased and suffers from the concept of the evaluator effect. 
There is simply no guarantee that all anomalies regarding usability problems have been extracted from the data.
This lack of a formal description of how user-experience problems unfolds it self on a physiological level over time, makes any definition of how to detect these anomalies regarding a UEP an educated guess from the researchers.
Due to limitations and the scope of this study, we will not attempt to formalize the problems outlined above, we do however outline the definitions used in this paper to describe and detect UEPs.

\subsection{Relation between psychology and physiology}
In this section, we will try and define what user experience problems are and how they might manifest themselves in
physiological measurements. This will be based partly on related research, our previous work and by analyzing data
collected from our experiments.

Before defining how we interpret user experience problems, let us remind ourselves how usability problems can be
defined. Lavery D. et al.~\cite[p. 254]{comp-eval-methods} suggests the following definition:

\begin{quotation}
  \textit{``A usability problem is an aspect of the system and/or a demand on the user which makes it unpleasant,
    inefficient, onerous or impossible for the user to achieve their goals in typical usage situations''} 
\end{quotation}

From this definition, we find it reasonable to expect test participants to experience affective states such as
\textit{frustration, irritation} or \textit{stress}. We also find this similar to suggestions by Brunn et
al.~\cite{LH-paper} and Lazar et al.~\cite{frustration_with_computers}, with \textit{frustration} being a predominant
affective state. Similarly, Ceaparu et al.~\cite{determining-causes-end-user-frust} claim that \textit{frustration} is
universally experienced by computer users, whenever expectations are not fulfilled and unexpected time delays are
imposed. However, how such affective states manifests themselves in physiological measurements, we deem to be an
advanced and complex topic on its own.

As mentioned by Liapis et al.~\cite{fusion4}, Cacioppo et al.~\cite[p. 8-9]{handbook-psychophysiology} offers five
plausible connections, or relations, between the psychological and physiological domains, briefly listed as:

\begin{itemize}[noitemsep, nolistsep]
\item \textbf{one-one:} a psychological event is associated with one and only on physiological event, and vice versa
\item \textbf{one-many:} a psychological event is associated with several physiological events. The opposite could also
  hold true, that several psychological events could cause the same physiological event.
\item \textbf{many-many:} a particular set of psychological events are related to a particular set of physiological
  events
\item \textbf{null:} that there exists no particular relation between physiological and psychological events.
\end{itemize}

According to Cacioppo et al., is could even be the case that one relation holds true for one individual, while a
different relation holds for another. We suspect the above to be one of the reason why we have been unable to find
definitive and formalized methods that identify affective state based on physiology. That is not to say that related
research have not been successful in doing so - quite the contrary - but only that results vary, methods are disparate
and varies depending on the context. We therefore find ourselves in a situation where an explorative approach is most
promising, drawing inspiration from related research.
\todo{list the sources/cites of a few studies with good results w.r.t. identify affective state}

If the \textbf{one-one} relation from those suggested by Cacioppo et at. holds, then some particular affective state
would always manifests itself in the same physiological event, e.g. fear would always see the same increase in
heart-rate, change in skin conductance, and the same patterns in brain-waves. We find this view unplausible, at least in
our context, and are more inclined to accept the premise that a \textbf{one-many} relation exists. We see this as
meaning that a particular psychological event can manifest itself into several different physiological events. However,
we do expect to see similarities in those physiological events, if originating from the same psychological event, e.g
feeling \textit{fear} leads to some increase in heart-rate, variance in skin conductance, and activity from particular
areas of the brain (specific sensors on the EEG). In this work, we also assume the above to be the case across all
individuals participating in our experiments.

\subsection{User experience problem definition}
Questions such as ``when does a particular affective state manifest itself'' and ``for how long...'' when considering a
continuous stream of physiological measurements from various different types of sensors, are difficult to answer. Not
only are they difficult to answer, but the answer might also vary significantly from person to person. Our previous work
was faced with with similar questions, however the problem was easier to approach since stimuli was very specific and
induced at very specific moments, leading us to approach the above differently.
\todo{cite our previous work somehow}

In this definition, we consider particular moments within the experiment to be \textit{normal}, where we expect test
participants to exhibit physiological states that are \textit{not} related to \textit{frustration} or similar negative
affective states. We refer to such \textit{normal} states as the \textit{baseline}. Further, we do not expect test
participants to exhibit any positive affective states during the experiment, such as \textit{happiness} or \textit{joy}
- our test setup is designed with this in mind. Based on these assumptions, we hypothesize that any physiological
measurements that deviate from \textit{normal} could be interpreted as negative affective state. This of course leads to
the question of how do we consider some particular measurements to be outside of this \textit{normal} state? The answer
depends on the particular sensor in question, and for this we lean on our previous work.

We consider affective state changes in GSR to manifest themselves 2-4 seconds after the experience occurred which
induced the stimuli, and is noticeable within a time-frame of 3-4 seconds thereafter. EEG is considerably different,
manifesting after 350 milliseconds, lasting 760 milliseconds. Reactions to stimuli can be seen manifesting in heart-rates
after 4 seconds, and lasts for 3 seconds. Lastly face changes has a delay of 500ms and lasts for 500ms.\todo{just pick cites from paper1}

Based on the above, we consider \textit{windows}, i.e. samples of continuous measurements within those boundaries, for
each sensor, for a particular moment in time. We do so for moments within what we expect to be \textit{normal} state and
likewise for \textit{unknown} moments. If those \textit{unknown} moments deviate from \textit{normal}, we expect that to
indicate that the test participant is \textit{frustrated}. The above is based on many assumption, both about human
physiology and how experiencing usability problems manifests themselves in said physiology. This is necessary in order
to formalize a method with which we can conduct experiments.

In our setup, the particular moments which we deem to be \textit{normal} are defined as the two first tasks the test
participant is asked to solve, i.e. all physiological measurements collected within those moments are treated as a
\textit{baseline}. This is because the two first tasks are always chosen from a set of tasks containing no seeded
usability problems. From the baseline, we extract various statistical measurements from which \textit{unknown} data
should deviate from, in order to signify the existence of a usability problem.

\todo{I think we need a section/table below this, where we concretise the above}
