%!TEX root = ..\Main.tex
\section{Method}
In this section, we explain the methods and practices we apply in order to answer our hypothesis. In particular, an
experiment was conducted and its setup will be explained, along with applied MI and statistical techniques used to
analyse physiological data collected during experiments.

\subsection{Problem-seeded test application}
In order to control the kind of seeded problems, and at which moments they should be revealed, we developed a software
application, into which we could embed such problems. This application is mimicking the functionality of a subset of
features found in ``real-world'' equivalent applications, but with controllable seeded problems.  Many choices could be
made as to the kind of software application it should mimic, but research shows that some domains within software
applications are more likely to induce stress and frustration in the user, compared to other domains.  As mentioned
under Related Work, Lazar et al.~\cite{frustration_with_computers} found that email- and text-related work tends to
induce the highest amounts of frustration in users. We use this discovery as inspiration to develop a ``mock'' email
application, i.e it does not send any emails but merely pretends to do so.

The application was built upon a self developed framework which facilitated seeding usability problems and creating a set of tasks for the user to complete. 
The application is kept simple with few features to decrease the risk of unintentionally introducing usability problems, while still having the basic functionalities of a normal email application.  
The UPs associated with a specific task were only active when the task was active, i.e. the test participant was
attempting to complete that particular task.
For instance a particular task involved attaching an image to an email. 
While the task is active, the attachment process will fail three times and be successful on the fourth. 
Had any other task been active the attachment process would not contain the seeded usability problem.
The program has a total of 11 tasks of which 7 contained a seeded usability problem. 
All tasks were randomized for each test participant besides the first two which contained no seeded problems, with the
intention of using them as baseline training data for novelty detection. 
Each task and their associated usability problem can be seen in Table~\ref{tab:ups-desc}.
Task 3 and 7 can be considered full stops, and are incompletable.
Tasks 1,2,4,5 and 6 can be completed given persistence or explorative user behavior, e.g. task 6 will in fact remove the contact, but to see it has done so requires the user to close the window and reopen it to validate it.

\begin{table}[h]
  \centering
  \begin{tabular}[c]{|p{60pt}|p{80pt}|p{80pt}|}
    \hline
    task name                         & description                                                                                                                    & seeded problem                                                                                           \\ \hline
    \small{1. Add~attachement}        & \small{Add an attachment to a mail}                                                                                            & \small{Program ``hangs'' for 2 seconds three times, before the attachment can be completed.}             \\ \hline
    \small{2. Add~contact}            & \small{Add a new contact to the contacts catalogue}                                                                            & \small{The ``Add Contact'' button will not work for the first three clicks}                              \\ \hline
    \small{3. Send~Draft}             & \small{Find a draft, either by creating a mail and drafting it or selecting a pre-created draft, and send it}                  & \small{An exception will show when they try to open the draft, making it impossible to send}             \\ \hline
    \small{4. Create~a~draft}         & \small{Create a draft with the body: ``Rød grød med føde''}                                                                    & \small{The keyboard layout changes to American, making it impossible to type the Danish character ``ø''} \\ \hline
    \small{5. Write~a~mail}           & \small{Create a mail with the body: ``Hi, my name is x and I am participating in a usability test''}                           & \small{At random intervals the caret will move while writing the mail}                                   \\ \hline
    \small{6. Remove~Contact}         & \small{Remove a specific contact from the contacts catalogue}                                                                  & \small{When clicking ``Delete'', the entire window will change to a black box}                           \\ \hline
    \small{7. Write~mail~2}           & \small{Write a mail with the body text ``Hello, I am having a birthday party 10 days from now, and this is your invitation!''} & \small{The window for writing a mail is unavailable, and the title changes to ``Not responding...''}     \\ \hline
    \small{8. Send~a~mail}            & \small{Send a mail with any text, to two contacts}                                                                             & \small{None}                                                                                             \\ \hline
    \small{9. Save~a~draft}           & \small{Create a mail, and draft it}                                                                                            & \small{None}                                                                                             \\ \hline
    \small{10. Reply~to~mail}         & \small{Reply to a mail}                                                                                                        & \small{None}                                                                                             \\ \hline
    \small{11. Write~and delete~mail} & \small{Write a mail containing any text, draft it and then delete it}                                                          & \small{None}                                                                                             \\ \hline
  \end{tabular}
  \caption{Usability problems descriptions}
  \label{tab:ups-desc}
\end{table}

\subsection{Experiment}
Our experiment is a traditional usability test(UT) setup: a software application is tested by
a test participant, solving pre-defined tasks. However, unlike a traditional UT setup, we attach physiological sensors
onto the test participant, and expose them to a software application that - intentionally but unknown to the test
participant - has usability problems embedded into it. This is to induce the anticipated affective state changes using
usability problems at \textit{known} moments during the experiment and to increase reproducibility of our
experiment. Furthermore, no usability experts are present in the room and no communication between the test participant
and usability expert takes place, while the usability test is undergoing. The experiment is set up and conducted inside
a traditional usability lab, located at Aalborg University~\cite{usability_lab_cassiopeia}.

\subsubsection{Hardware}
The hardware used for the experiment is an Emotiv Epoc~\cite{emotiv_epoc_website} for Electroencephalograph (EEG) to recording brain activity, a Mindplace Thoughtstream~\cite{thoughtstream} for Galvanic Skin Response (GSR), an Arduino with a pulse-sensor~\cite{pulsesensor} with modified software~\cite{pulsesensorgit} to measure heart rate (HR) and a Kinect V2\cite{kinect_specs3} for tracking facial changes.
The pulse-sensor software was modified to send beats per minute (BPM), inter-beat interval (IBI) and raw signal every 20 ms.
All devices are low-cost consumer grade hardware.

\subsubsection{Participants and setup}
A total of 29 people participated in the test (16 male, age 20-29 SD 2.24, 13 female age 19-32 SD 3.04).
The participants were students recruited from multiple faculties at Aalborg University. 
All participants filled out a Big-Five\cite{big5} which revealed no bias in terms of personality.
\todo{We should probably add numbers here}

The tests were conducted from the 13th of April, 2016, to the 30th of April 2016 and from 8:00 to 16:00 every day. 
The participants were asked to fill out a consent form prior to participating. 
The participants were instructed in how to use the test program which included how to see the tasks, and how to indicate whether or not they could complete the given task. 
Before starting the test all hardware was attached to the participant and verified in terms of connectivity. 
The EEG was connected to the head according to the 10-20 system\cite{eeg_tech_10_20}, the GSR and pulse sensor was attached to their non-dominant hand. 

\subsubsection{Procedure}
The test starts with an application the mail client simulation.
In the bottom right corner of the application a ``task'' window is located, which contains information regarding the test participant's current task. 
A red and green button indicated that you have ``completed'' or ``not completed'' a task.  
Each task had an indefinite time threshold, but the participant could choose to continue to the next task at any given moment by pressing either a green or a red button on the keyboard. 
The participants were not allowed to communicate with the test conductor during the test, unless absolutely necessary. 
In case any communication had to take place, it was done through a microphone and speakers.
\todo{really, a flowchart showing the process might be a good idea}

\subsection{Usability problem detection}
Before predicting potential usability problems, we establish certain criteria to be fulfilled in order for them to
classified as such.  We consider a particular section within the experiment to be \textit{normal}, i.e. containing no
usability problems.  We do not expect test participants to exhibit physiological states that are related to
\textit{frustration} or similar negative affective states during this section. We refer to this normal section as the
\textit{baseline}. The baseline is established in the beginning of the test during the two first tasks, after the
initial relaxation section, and before the third task begins. This is because the two first tasks are always chosen from
a set of tasks containing no seeded usability problems. All physiological measurements collected within this section are
treated as a \textit{baseline} data.  From the baseline, we extract various statistical measurements from which
\textit{unknown} data should deviate from, in order to signify the existence of a potential usability problem.

\subsubsection{Categorizing seeded problems}
write this
\todo{write something about: since the seeded problems are different in nature, we also expect them to trigger different
physiological reactions and at different moments}

\subsubsection{Detecting affective state changes}
We consider affective state changes in GSR to manifest themselves 2-4 seconds after the experience occurred which
induced the stimuli, and is noticeable within a time-frame of 3-4 seconds thereafter. EEG is considerably different,
manifesting after 350 milliseconds, lasting 760 milliseconds. Reactions to stimuli can be seen manifesting in heart-rates
after 4 seconds, and lasts for 3 seconds. Lastly face changes has a delay of 500ms and lasts for 500ms. As mentioned,
the above is based on our previous work, which in turn is based on related research within HCI, psychology and
physiology of the human body~\cite{9th_semester_project}.
\todo{cite also 1st. paper and lots of others. Put them in the beginning of this paragraph}

Based on the above, we consider \textit{windows}, i.e. samples of continuous measurements within those boundaries, for
each sensor, for a particular moment in time. We do so for moments within what we expect to be \textit{normal} state and
likewise for \textit{unknown} moments. If those \textit{unknown} moments deviate from \textit{normal}, we expect that to
indicate that the test participant is \textit{frustrated}.

\todo{write more here about how we detect stuff}