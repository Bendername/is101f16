%!TEX root = ..\Main.tex
\section{Results}
The data is calculated for 35 of the test participants, as 4 were removed due to faulty data on one or more sensors.	
A total of x. events were encountered and the average length of a test was x.
The ratio between area designated to containing UPs, and areas that do not contain UPs, in on average 0.37, i.e. a
little less than 2/3 of the data on which we attempt to predict UPs is \textit{normal} data. The implication of this is
that if a classifier were to distribute random guesses based on our data, about 63\% of them would fall within normal
data, and 37\% within data that contains UPs, slightly shifting the difficulty to our disadvantage.
\todo{lav std dev over det her!}
% hvordan her vi udregnet det?
%Random ratio - 37,52 
Graph x.x.x and. x shows the scatter plot of EHR and FCR for the different sensors. 
Graph x.x.x and. x shows the scatter plot of precision and FCR for the different sensors. Where the average for the different Nu-values is the bordered circles.

\newcommand{\fuckinggraph}[4]{
    \begin{figure}[h!]
    \begin{minipage}[t]{0.5\textwidth}
        \includegraphics[width=\linewidth,keepaspectratio=true]{graphics/graphs/#1/#2}
        \caption{#3}
        \label{#4}
    \end{minipage}
    \end{figure}
}

\newcommand{\fuckinggraphevenidontwanttorepeatmyself}[4]{ %short sensor caption1 label1
  \fuckinggraph{#1}
  {False_cover_rate_(FCR)-Events_hit_rate_(EHR)-CovNu-#2.pdf}{#3}{#4}
  %{False_cover_rate_(FCR)-Precision-CovNu-#2.pdf}{#5}{#6}
}

\fuckinggraphevenidontwanttorepeatmyself{short}{GSR}
{GSR showing events hit percent and unwanted area covered. Nu value shade: 0 = green, 1 = red}{fig:gsr_event_ehr}
%{GSR showing precision and unwanted area covered}         {lab:gsr_pres_ehr}

\fuckinggraphevenidontwanttorepeatmyself{short}{EEG}
{EEG showing events hit percent and unwanted area covered. Nu value shade: 0 = green, 1 = red}{fig:eeg_event_ehr}
%{EEG showing precision and unwanted area covered}         {lab:eeg_pres_ehr}

\fuckinggraphevenidontwanttorepeatmyself{short}{HR}
{Heart rate showing events hit percent and unwanted area covered. Nu value shade: 0 = green, 1 = red}{fig:hr_event_ehr}
%{Heart rate showing precision and unwanted area covered}         {fig:hr_pres_ehr}

\fuckinggraphevenidontwanttorepeatmyself{short}{FACE}
{Kinect showing events hit percent and unwanted area covered. Nu value shade: 0 = green, 1 = red}{fig:face_event_ehr}
%{Kinect showing precision and unwanted area covered}         {lab:face_pres_ehr}

\textbf{Sensors likeness and differences}\\
As seen in Figure x,x,x and x, the Kinect and EEG machines shows precision above random guessing at low Nu-values and,
as expected, closes in on the random guessing (37,52\%), as we increase the Nu-value. However, as seen in Table
\ref{TABLE] avg_stats_sensors}, the GSR only slightly excite the random ratio at some of the values. Meaning that
to have a reasonable precision with few false positives, a lower Nu-value would be desirable.
These findings indicate that the results from the Kinect and EEG sensors are presumably not related to random guessing, but can give a moderately qualified answer to where usability problems can be found, and the GSR show some of the same tendencies.
While the HR values does not excite the threshold for random guessing, it still shows encouraging results in regards of hitting as many different events as possible while keeping a low FCR, as seen in Figure~\ref{fig:hr_event_ehr}.

\begin{table}[H]
  \centering
  \textbf{GSR}\vspace{2pt}
  \begin{tabularx}{\columnwidth}{cXXc}
    \toprule
    \textbf{Nu} & \textbf{Precision} & \textbf{EHR} & \textbf{FCR} \\
    \midrule
    1.0\% & 30.0\% & 4.5\% & 2.1\% \\ \hline
    5.0\% & 36.4\% & 14.5\% & 7.2\% \\ \hline
    25.0\% & 39.0\% & 36.6\% & 23.5\% \\ \hline
    50.0\% & 37.8\% & 63.0\% & 44.8\% \\ \hline
    75.0\% & 37.6\% & 84.0\% & 70.6\% \\ \hline
    100.0\% & 37.1\% & 99.8\% & 99.8\% \\ \hline
    \bottomrule
  \end{tabularx}

  \vspace{4pt}

  \textbf{EEG}\vspace{2pt}
  \begin{tabularx}{\columnwidth}{cXXc}
    \toprule
    \textbf{Nu} & \textbf{Precision} & \textbf{EHR} & \textbf{FCR} \\
    \midrule
    1.0\% & 43.7\% & 23.8\% & 18.5\% \\ \hline
    5.0\% & 38.6\% & 35.1\% & 20.9\% \\ \hline
    25.0\% & 35.0\% & 68.5\% & 61.8\% \\ \hline
    50.0\% & 36.7\% & 90.3\% & 86.1\% \\ \hline
    75.0\% & 37.0\% & 91.7\% & 90.8\% \\ \hline
    100.0\% & 37.2\% & 93.7\% & 95.8\% \\ \hline
    \bottomrule
  \end{tabularx}

  \vspace{4pt}

  \textbf{Kinect}\vspace{2pt}
  \begin{tabularx}{\columnwidth}{cXXc}
    \toprule
    \textbf{Nu} & \textbf{Precision} & \textbf{EHR} & \textbf{FCR} \\
    \midrule
    1.0\% & 51.7\% & 21.7\% & 10.0\% \\ \hline
    5.0\% & 50.8\% & 33.1\% & 28.5\% \\ \hline
    25.0\% & 44.1\% & 64.8\% & 57.4\% \\ \hline
    50.0\% & 40.5\% & 89.0\% & 75.7\% \\ \hline
    75.0\% & 37.5\% & 98.2\% & 92.5\% \\ \hline
    100.0\% & 36.5\% & 99.6\% & 96.2\% \\ \hline
    \bottomrule
  \end{tabularx}

  \vspace{4pt}

  \textbf{Heart rate}\vspace{2pt}
  \begin{tabularx}{\columnwidth}{cXXc}
    \toprule
    \textbf{Nu} & \textbf{Precision} & \textbf{EHR} & \textbf{FCR} \\
    \midrule
    1.0\% & 31.4\% & 10.1\% & 4.3\% \\ \hline
    5.0\% & 36.0\% & 28.5\% & 15.1\% \\ \hline
    25.0\% & 36.4\% & 61.5\% & 45.5\% \\ \hline
    50.0\% & 35.6\% & 87.1\% & 67.2\% \\ \hline
    75.0\% & 36.1\% & 96.8\% & 86.1\% \\ \hline
    100.0\% & 36.7\% & 99.8\% & 99.3\% \\ \hline
    \bottomrule
  \end{tabularx}

  \caption{Average statistics for each sensor}
  \label{[TABLE] avg_stats_sensors}
\end{table}
\begin{table}[h]
  \centering
  \textbf{Conservative Approach}\vspace{2pt}
  \begin{tabularx}{\columnwidth}{cXXc}
    \toprule
    \textbf{Votes} & \textbf{Precision} & \textbf{EHR} & \textbf{FCR} \\
    \midrule
    1 & 40.0\% & 54.0\% & 42.8\% \\ \hline
    2 & 40.7\% & 23.2\% & 12.7\% \\ \hline
    3 & 33.6\% & 5.8\% & 1.9\% \\ \hline
    4 & 4.3\% & 0.5\% & 0.1\% \\ \hline
    \bottomrule
  \end{tabularx}

  \vspace{4pt}

  \textbf{Aggressive Approach}\vspace{2pt}
  \begin{tabularx}{\columnwidth}{cXXc}
    \toprule
    \textbf{Votes} & \textbf{Precision} & \textbf{EHR} & \textbf{FCR} \\
    \midrule
    1 & 37.6\% & 98.8\% & 91.9\% \\ \hline
    2 & 39.2\% & 92.2\% & 78.7\% \\ \hline
    3 & 41.4\% & 72.1\% & 58.0\% \\ \hline
    4 & 35.4\% & 33.7\% & 25.3\% \\ \hline
    \bottomrule
  \end{tabularx}

  \caption{Average statistics for voting}
  \label{[TABLE] avg_stats_voting}
\end{table}
\newcommand{\votinggraphs}[6]{
    \begin{figure}[h!]
    \begin{minipage}[t]{0.5\textwidth}
        \includegraphics[width=\linewidth,keepaspectratio=true]{graphics/graphs/voting/#1}
        \caption{#2}
        \label{#3}
    \end{minipage}
    \hspace*{\fill} % it's important not to leave blank lines before and after this command
    \begin{minipage}[t]{0.5\textwidth}
        \includegraphics[width=\linewidth,keepaspectratio=true]{graphics/graphs/voting/#4}
        \caption{#5}
        \label{#6}
    \end{minipage}
    \end{figure}
}

\newcommand{\easyvotinggraphs}[5]{ %y_axis caption1 label1 caption2 label2
  \votinggraphs
  {voting-False_cover_rate_(FCR)-#1-Aggressive.pdf}{#2}{#3}
  {voting-False_cover_rate_(FCR)-#1-Conservative.pdf}{#4}{#5}
}


\easyvotinggraphs{Events_hit_rate_(EHR)}
{Showing voting based on an aggressive scoring function. The lightest blue shade is 1 vote, and the darkest is 4 votes. The two shades in between are voting 2 and 3.}{fig:voting_aggresive_ehr}
{Showing voting based on a conservative scoring function. The lightest blue shade is 1 vote, and the darkest is 4 votes. The two shades in between are voting 2 and 3.}{fig:voting_conservative_ehr}


%\easyvotinggraphs{Precision}
%{Showing voting based on an aggressive scoring function}{fig:voting_aggressive_pres}
%{Showing voting based on a conservative scoring funtion}{fig:voting_conservative_pres}

Looking at Figure~\ref{fig:gsr_event_ehr}, \ref{fig:eeg_event_ehr}, \ref{fig:hr_event_ehr} and \ref{fig:face_event_ehr},
as well as Table~\ref{[TABLE] avg_stats_sensors}, it can be seen that choosing a higher Nu-value for your classifier can yield interesting propositions if the classifier should cover as many problems as possible while minimizing the area wrongly covered.
In other words, if one's aim is to hit all events, a high Nu-value must be chosen, and it comes with the trade-off of
more anomalies being placed outside events. 
While the GSR and the HR both have a smooth curve through the averages of which indicates a stable classifier, the Kinect seems to be more unstable in its relation between EHR and FCR. However, Kinect seems to regain some of its stability with higher Nu-values. As shown in~\ref{TABLE] avg_stats_sensors} and seen in Figure x, the EEG deviate from the others by already at low Nu-values doing a very aggressive prediction and predicting x amount more in average than the X sensor.

All the graph reveals that across all the test participants, no golden Nu-value, when trying to detect usability problems, is present, but they also show that sensors show signs of being able to detect when a test participant encounters a usability problems.
