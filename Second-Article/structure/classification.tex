\section{Classification}
To predict whether or not a usability problem is present a one-class SVM has been used.\todo{LIBSVM}
The one-class SVM is an algorithm used for novelty detection. Novelty detection means that the training data contains no anomalies, which is a requirement we should meet by training on the data from the first two tasks\todo{og måske resting dataen?}
The rest of the data will then act as the prediction set.

We create one one-class SVM for each of the sensors. 

\subsection{Data processing}
For the one-class SVM can work with the data, the data first has to be transformed into feature vectors.
A feature vector is simply a n-space vector containing values extract from the a sensor.
The features selected are based off .. \todo{Old article?}
A feature vector is created for each 100~ms in a test, which for some of the sensors result in data overlapping\todo{Spørg Thomas om hvordan vi lige skriver det her med data overlapping}.



\subsection{Prediction}
After the model for the one-class SVM is trained on the two first task, the model can then be used to predict on feature vectors from the remaining task. 
The return value from this prediction is a binary 1 for a normality and -1 for an anomaly.
The fusion of the results from the different machines will be a simple voting approach where a machine must declare whether it has found an anomaly to the  given point in time. If x\todo{How many must agree?} sensors agrees the given point will be flagged as an anomaly. 
