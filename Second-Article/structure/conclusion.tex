%!TEX root = ..\Main.tex

\section{Conclusion}

\subsection{Future work}
One of the most promising areas to continue research within user experience problem detection through physiological affective state, is the means to detect the affective state change. 
We have chosen off-the-shelf standard and basic solutions to solve the problem, but doing research aimed at optimizing these techniques should be done. 
There have been shown promising results using a 1-class SVM as detection method, as well as neural networks have been used.
Further, the entire study has revolved around fusing multiple sensors, and while one method may work particularly well with a GSR, it may prove worse or even bad to use with an EEG. 
A further exploration into the feature space could also be conducted. 
Especially for the EEG better hardware could open up for some more interesting models like DASM12\cite{eeg_music_listening}. 

Further research also has to be done with regards to the emotional baggage people arrive with. 
The personality type of a user fundamentally change the way their physiological response is, such as the difference between introvert and extrovert GSR streams. 
It is natural to hypothesize that other emotional contextual influences such as being hungry or having ``a bad day'' may influence the physiological patterns as well. 
