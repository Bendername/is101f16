%!TEX root = ..\Main.tex

\section{Conclusion}

\subsection{Future work}
One of the most promising areas to continue research within user experience problem detection through physiological affective state, is the means to detect the affective state change. 
We have chosen off-the-shelf standard and basic solutions to solve the problem, but doing research aimed at optimizing these techniques should be done. 
There have been shown promising results using a 1-class SVM as detection method, as well as neural networks have been used.
Further, the entire study has revolved around fusing multiple sensors, and while one method may work particularly well with a GSR, it may prove worse or even bad to use with an EEG. 
A further exploration into the feature space could also be conducted. 
Especially for the EEG better hardware could open up for some more interesting models like DASM12\cite{eeg_music_listening}. 

Further research also has to be done with regards to the emotional baggage people arrive with. 
The personality type of a user fundamentally change the way their physiological response is, such as the difference between introvert and extrovert GSR streams. 
It is natural to hypothesize that other emotional contextual influences such as being hungry or having ``a bad day'' may influence the physiological patterns as well. 

A fundemental issue with the results of the paper also includes the actual definition of frustration in terms of usability problems. 
While there is a lot of litterature on this subject, almost no one has explored the actual development of frustration on a physiological level. 
We have set up somewhat naive conditions because it is beyond the scope of this study to examine and map the physiological changes over time for frustrating events. 
It may be that physiological reactions increase in amplitude over time as the frustrating event persists with spikes within this period of time. This could have potential implications for the amount of outliers detected, both as false-negative and false-positives. If a frustrating event is seen as such, the initial slope of frustration would not be detected by our anomaly detection, this could potentially be done with a more proper model. 
The fact that this area is largely unexplored leaves a lot of assumptions to be made, and as such the validity of our results can be questioned. 
Further research in all sensors in terms of making a general model of a physiological representation of frustration would greatly increase the replicability of research results, and create a common path to iterate over research to come up with better models and results. 
This is, however, a huge task even just for one sensor and within subject, but it is the most lacking area of research within this area of HCI currently.
