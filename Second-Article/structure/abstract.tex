%!TEX root = ..\Main.tex

\begin{abstract}

\textbf{Objectives:}
Traditional usability testing focuses on performance metics such as task completion time and effort required. 
Further it requires third-party expert subjective evaluations to estimate problems, and label their severity.
This paper attempts to take an approach were usability problems are detected from a users affective state, using physiological sensors and machine learning.
\textbf{Methods:}
A self developed email client, and usability problems were deliberately seeded into it. 35 test participants (18 male,
17 female) had to solve 11 tasks, of which 7 had usability problems of varying severity.
Novelty detection was used to find affective state outliers, using a one-class SVM as classfier.
\textbf{Results:}
Average case classification result did not yield results much better than random guessing, due high variance in results. 
However, promising results were found when considering the five best, and circumstances as to why this is, is discussed.
\textbf{Conclusion:}
We explored the idea that it was possible to find usability problems from a test participants affective state. 
It was possible, but the more aggresive the classifier was tuned, the more noise, i.e. false-positives, was included in the prediction result set. 
\end{abstract}
