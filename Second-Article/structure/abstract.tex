%!TEX root = ..\Main.tex

\begin{abstract}


\textbf{Objectives:}
Usability testing has always focused on performance metics such as task completion time and effort required. 
Further it requires third-party expert evaluations to estimate problems, and label their severity.
This paper attempts to take an approach were usability problems are evaluated from a users affective state, using physiological sensors and machine learning.
\textbf{Methods:}
A self developed email client was used with seeded problems. 35 test participants (18 male, 17 female) had to solve 11 tasks, of which 7 had usability problems with different severity.
Novelty detection was used to find affective state outliers where a one-class SVM was used as classfier.
\textbf{Results:}
It was possible to find some usability errors from the test participants affective state, but as the amount increased so did the amount of noise classified as well.
\textbf{Conclusion:}
We explored the idea that it was possible to find usability problems from a test participants affective state. 
It was possible, but the more aggresive the classifier was tuned, the more noise was included in the prediction result set. 
\end{abstract}
