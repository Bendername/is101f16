%!TEX root = ..\Main.tex

\begin{abstract}


\textbf{Objectives:}
Usability testing has always focused on performance metics such as task completion time and effort required. 
Further it requires third-party expert evaluations to estimate problems, and label their severity.
This paper attempts to take an approach were User Experience is evaluated, by finding user experience problems from a users affective state, using physiological sensors and machine learning.
\textbf{Methods:}
A self developed email client was used with seeded problems. 36 test participants (19 male, 17 female) 11 tasks, of which 7 had usability problems.
Anomaly detection was used to find affective state outliers.
\textbf{Results:}
\textbf{Conclusion:}
\end{abstract}
