\section{Hypothesis}
The focus of this paper is to use multiple consumer graded hardware sensors to detect usability problems. 
By using multiple sensors and appropriate machine intelligence techniques we hope to show that it is possible to use objective physiological data to find usability problems through a users effective state as a direct alternative or extra tool to traditional usability evaluation methods. 
As such, our hypothesis' are:\\\\

\newcommand{\hypo}[2]{%
  \textbf{H#1:} \textit{#2} \\
}
%\textit{\textbf{H.1.} Physiological sensor data can be used to detect usability problems with a significantly low amount of false-positives. False-positives are the classification of a usability problem being present, while it is not.}\\\\
%\textit{\textbf{H.2.} Physiological sensor data can be used to detect usability problems with
% a significantly low amount of false-negatives. False-negatives are the failure to classify a usability problem being present}\\\\
%\textit{\textbf{H.3.} There is a statistical significant correlation between severity ratings and physiological measurements.}

\hypo{1}{Using consumer-graded physiological sensor-hardware, it is possible with a significant degree
  of certainty\todo{Tror ikke det er muligt at sige her, med mindre vi definere hvad vi skal sætte det op i mod, vi kunne måske finde et benchmark? som i de 85\% fra den anden artikel}, to predict moments where usability problems are experienced during a usability test. }

\hypo{1a}{Using physiological data from multiple sensors increase the degree of certainty to which moments of usability problems are predicted}

\hypo{1b}{Using physiological data from multiple sensors decrease the likelihood to which false-positives occur.}\todo{Kan vi godt prøve at se på, men vi skal have defineret hvad vi mener med en false positiv}

\hypo{2}{Usability problems of higher severity are more likely to be correctly predicted as a problem, compared to
moments containing usability problems of lower severity}

%http://www.tandfonline.com.zorac.aub.aau.dk/doi/pdf/10.1080/03610730500206808
% Cite for SAM

%http://www.robots.ox.ac.uk/~davidc/pubs/NDreview2014.pdf
% Cite and use for novelty detection
