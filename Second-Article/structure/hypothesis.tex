\section{Research Question}
The focus of this paper is to explore the possibility of using physiological data within usability testing. We argue
that this is a relatively unexplored territory within HCI - at least no predominant or popular methods seem to
exist. Since research within the area is still active, we take this to indicate an interest in finding such a method -
we want to contribute to this search.

We propose a method using physiological data directly from participants during testing.  Multiple consumer grade
sensors are used to gather physiological data on which machine learning techniques are applied to predict changes in the
affective state of the user and from that determine if a usability problem is present in the system under test. The study is
explorative in nature because of the limited research already conducted within this field.  The focus will be to
identify specific areas of interest for researchers in terms of using sensors and affective state changes to find
usability problems.

Using the above approach, we state the following questions:

\textit{Is it possible to detect usability problems from physiological data gathered during testing?}

\textit{Could a combination of physiological data gathered from multiple sensors possibly increase the reliability of such detections?}

Due to the explorative nature of this study, we do not expect to be able to give definitive answers to the above, but
rather provide valuable insights and discussions into how different approaches to analyzing the data can give different results.