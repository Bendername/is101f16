\section{Hypothesis}
The focus of this paper is to use multiple consumer graded hardware sensors to detect usability problems. 
By using multiple sensors and appropriate machine intelligence techniques we hope to show that it is possible to use objective physiological data to find usability problems through a users affective state as a direct alternative or extra tool to traditional usability evaluation methods. 
This is under the assumption that we \textit{can} predict frustration and thereby usability problems.
We outline the method we use to determine the extend to which we can detect usability problems in \ref{sec:method}.
As such, our hypothesis' are:\\\\

\newcommand{\hypo}[2]{%
  \textbf{H#1:} \textit{#2} \\
}
%\textit{\textbf{H.1.} Physiological sensor data can be used to detect usability problems with a significantly low amount of false-positives. False-positives are the classification of a usability problem being present, while it is not.}\\\\
%\textit{\textbf{H.2.} Physiological sensor data can be used to detect usability problems with
% a significantly low amount of false-negatives. False-negatives are the failure to classify a usability problem being present}\\\\
%\textit{\textbf{H.3.} There is a statistical significant correlation between severity ratings and physiological measurements.}
\hypo{1}{There is a statistically significant improvement using sensor fusion to detect usability problems compared to using individual sensors.}
%http://www.tandfonline.com.zorac.aub.aau.dk/doi/pdf/10.1080/03610730500206808
% Cite for SAM

%http://www.robots.ox.ac.uk/~davidc/pubs/NDreview2014.pdf
% Cite and use for novelty detection



We measure and compare detecting usability problems using individual sensors, with using multiple sensors. According to
our hypothesis, we expect multiple sensors to yield significantly more accurate predictions.
